\documentclass[a4paper, 12pt]{article}
\usepackage[left=2.5cm,
            right=2.5cm,
            top=2.5cm,
            bottom=2.5cm,
            bindingoffset=0cm]{geometry}
% \usepackage{array}
% \usepackage[indent=0pt]{parskip}
\usepackage{hyphsubst}
\usepackage{setspace}
\onehalfspacing

\usepackage{hyperref}
% \usepackage{float}
% \usepackage{graphicx}
% \graphicspath{ {./images/} }
% \usepackage{subfig}
% \usepackage{enumerate}
\usepackage[normalem]{ulem} % underlining
% \usepackage{booktabs} % tables
% \PassOptionsToPackage{table}{xcolor}% coloring tables

% LANGUAGE + FONT
		    
\usepackage[english]{babel}

% \usepackage{natbib}
% \renewcommand{\bibsection}{~\\\textbf{References}}
% \bibpunct[: ]{[}{]}{;}{a}{}{,}
% \bibliographystyle{rusnat}

\usepackage[style=authoryear]{biblatex}
\addbibresource{ref.bib}


\usepackage{fontspec}  
\setmainfont{Gentium Plus}

% DRAWING

\usepackage{tikz}
\usepackage{tikz-qtree}
\usetikzlibrary{shapes.geometric}
\usetikzlibrary{trees,arrows}
\usetikzlibrary{positioning}

% LINGUISTICS 

%\usepackage{gb4e}
\usepackage{expex}
\lingset{aboveglftskip=0ex, belowglpreambleskip=0ex, belowexskip=1ex, aboveexskip=1ex, interpartskip=0ex}

\gathertags
\usepackage[glossaries]{leipzig}
\newleipzig{cmpr}{cmpr}{Comparative}
\newleipzig{sprl}{sprl}{Superlative}
\newleipzig{indef}{indef}{Indefinite}
% \makeglossaries

\usepackage{stmaryrd}

% MATH
\usepackage{amssymb}

\begin{document}
\begin{sloppypar}

% -------------------------- текстиктекстиктекстик -----------------------------

\textbf{Partitive specificity and kind reference in Latvian}

There are significant parallels between individual and kind reference across languages, and it has been observed that kinds share properties with definite expressions \parencite[a.m.o.]{carlson1977referencekindsenglish, chierchia1998referencekindslanguages, krifka1995genericityintroduction}. E.g., English \textit{the} can refer to both individuals (\textit{the lion is roaring}) and kinds (\textit{the lion is extinct}).

Baltic languages offer further evidence. These languages employ a special suffix restricted to adjectives, often interpreted as conveying definiteness \parencite[160]{sereikaite2019strongvsweak,holvoet2012semanticmapdefinite,kalnaca2021latviangrammar}. Additionally, this marker is obligatory with classifying adjectives \parencite{rutkowski2006classifyingadjectivesnoun}, i. e., those that denote a subclass within the noun's class \parencite{mcnally2004relationaladjectivesproperties}. However, unlike the definite article, this marker remains obligatory even when the entire phrase doesn't refer to a kind. \parencite{sereikaite2017kindreferencedps} proposes that it is polysemous between \textsc{iota} and \textsc{down} operators, but compositionality questions arise.

This paper argues that the Latvian marker \textit{-ai-} conveys not definiteness, but rather partitive specificity \parencite{enc1991semanticsspecificity}, scoping under the adjective. In simpler terms, it requires the existence of a unique salient (plural) individual within the noun's extension, of which the DP's referent is a part.

\ex<>
    \begingl
        \glpreamble \{There were new shirts in a clothing store.\}//
        \gla Jānis sev nopirka sarkan-o kreklu//
        \glb Jānis himself bought red-\Def{}.\Acc{} shirt//
        \glft 'Jānis bought himself a red shirt.'//
    \endgl
\xe

This aligns with the conceptualization of kinds as pluralities of subkinds proposed in \parencite{krifka1995genericityintroduction}. Under this view, \textit{-ai-} can be assigned a unified interpretation in terms of partitive specificity. The analysis is formalized using dynamic semantics \parencite{heim1982semanticsdefiniteindefinite,dekker1996valuesvariablesdynamic} and extends to account the use of \textit{-ai-} with \textit{ad-hoc} kinds \parencite[82]{mendia2019referenceadhoc,holvoet2012semanticmapdefinite}. Furthermore, the paper explores the previously unregarded fact that \textit{-ai-} is used with classifying relational nouns as well, and the consequences of it for the analysis.

% \bibliography{../../ref}
\printbibliography

\end{sloppypar}
\end{document}
