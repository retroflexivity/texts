\documentclass[a4paper,12pt]{article}
\usepackage[left=2.5cm,
            right=2.5cm,
            top=2.5cm,
            bottom=2.5cm,
            bindingoffset=0cm]{geometry}
% \usepackage{array}
% \usepackage[indent=0pt]{parskip}
\setlength{\columnsep}{1cm}
\raggedbottom
\usepackage{hyphsubst}
\usepackage{setspace}
\onehalfspacing
\usepackage{parskip}
\setlength{\parindent}{0pt}

\usepackage{enumitem}
\setlist{nosep}

% \renewcommand{\thesection}{\arabic{section}.}
% \renewcommand{\thesubsection}{}
% \setlength{\columnsep}{1.6cm}
\usepackage{sectsty}
\sectionfont{\Large}
\subsectionfont{\large}
% \subsubsectionfont{\normalsize}
\usepackage{titlesec}
\titlespacing*{\section}{0pt}{3ex plus 0ex minus .2ex}{2ex plus .2ex}
% \titlespacing*{\subsection}
% {0pt}{2ex plus 0ex minus .2ex}{0ex plus .2ex}


\usepackage{hyperref}
% \usepackage{float}
% \usepackage{graphicx}
% \graphicspath{ {./images/} }
% \usepackage{subfig}
% \usepackage{enumerate}
\usepackage[normalem]{ulem} % underlining
% \usepackage{booktabs} % tables
% \PassOptionsToPackage{table}{xcolor}% coloring tables
\usepackage{csquotes}
\MakeOuterQuote{"}
% \MakeInnerQuote{'}
\usepackage[english]{babel}

% LINGUISTICS 

\usepackage{expex}
\lingset{aboveglftskip=0ex, belowglpreambleskip=0ex, belowexskip=0ex, aboveexskip=1.5ex, interpartskip=0ex}
\lingset{Everyex={\parskip=0pt}}
\gathertags

\let\expexgla\gla % dumb unicode-math conflict
\AtBeginDocument{\let\gla\expexgla}

\usepackage[glossaries]{leipzig}
\newleipzig{cmpr}{cmpr}{Comparative}
\newleipzig{sprl}{sprl}{Superlative}
\newleipzig{indef}{indef}{Indefinite}

% TYPOGRAPHY

\usepackage{amsmath}
\usepackage{stmaryrd}
% \usepackage{mathspec}
\usepackage{fontspec}
\usepackage{unicode-math}

\setmainfont{Libertinus Serif}
\setsansfont{Libertinus Sans}
\setmathfont{Libertinus Math}

\DeclareEmphSequence{\bfseries}

% BIBLIOGRAPHY		    

% \usepackage{natbib}
% \renewcommand{\bibsection}{~\\\textbf{References}}
% \bibpunct[: ]{[}{]}{;}{a}{}{,}
% \bibliographystyle{rusnat}

\usepackage[style=authoryear,url=false,doi=false,isbn=false,eprint=false,date=year]{biblatex}
\addbibresource{../../ref.bib}

% DRAWING

\usepackage{tikz}
\usepackage{tikz-qtree}
\usetikzlibrary{shapes.geometric}
\usetikzlibrary{trees,arrows}
\usetikzlibrary{positioning}


\begin{document}

% -------------------------- текстиктекстиктекстик -----------------------------

Arkady Shaldov, HSE, 29.05.2024

{\huge A proper name semantics for kinds}

\section{Introduction}

\subsection{Kind reference}

\emph{Kind reference} is the ability of NPs to refer to the intension of predicates as a whole. Diagnostic contexts are kind-level predicates \parencite{carlson1977referencekindsenglish}:

\begin{itemize}
    \item \textit{rare}
    \item \textit{inhabit}
    \item \textit{extinct}
    \item \textit{invent} (DO)
    \item etc.
\end{itemize}

Among the expressions that are usually considered to refer to kinds are

\begin{itemize}
    \item The kind-level singular \textit{the} \parencite{carlson1977referencekindsenglish}.
\end{itemize}

\pex<>
    \a<> The mammoth is extinct.
    \a<> The anteater eats ants.
\xe

\begin{itemize}
    \item DPs in the taxonomic reading (quantifying over subkinds) \parencite{dayal2004numbermarkingdefiniteness}.     
\end{itemize}    

\pex<clasdet>
    \a<some> Some anteaters are extinct.\trailingcitation{At least one of \{\textsc{giant ae., silky ae., \dots}\}}
    \a<thepl> The anteaters are divided into two families.\trailingcitation{All of \{\textsc{giant ae., silky ae., \dots}\}}
\xe

\begin{itemize}
    \item Bare plurals. Also note scopelessness \parencite{chierchia1998referencekindslanguages}.
\end{itemize}
    
\pex<>
    \a Huge angry anteaters inhabit South America.
    \a I didn't see huge angry anteaters.\trailingcitation{only \textsc{anteaters} > $\neg$, not $\neg$ > \textsc{anteaters}}
\xe
    
\begin{itemize}
    \item \textit{kind-of} \parencite{carlson1977referencekindsenglish}
\end{itemize}
    
\ex<>
    He is the kind of person that always speaks first.
\xe

The first two refer to \emph{well-established kinds} (WEK). The last two refer to \emph{\textit{ad-hoc} kinds} \parencite{mendia2019referenceadhoc}. This work only concerns the former.

\subsection{Classificatory adjectives}

The classificatory adjectives are the structurally lowest, semantically peculiar class of adjectives.
    
    \pex<>
        \a<> polar bear
        \a<> technical architect
        \a<> functional grammar
    \xe

\textcite{mcnally2004relationaladjectivesproperties} propose that they denote properties of kinds.

\subsection{What are kinds?}

It is unclear.

\begin{itemize}
    \item \textcite{carlson1977referencekindsenglish}: a special kind of entity.
    \item \textcite{chierchia1998referencekindslanguages} (the Neocarlsonian approach): a special kind of entity --- an intensional totality of individuals.
\end{itemize}

WEK-denoting nouns are considered to be \emph{proper name--like} expressions \parencite{carlson1977referencekindsenglish,krifka1995genericityintroduction}. They pass the \textit{so-called} test.

\ex<>
    The giant anteater is so called because it can be more than two meters long.
\xe

(The anaphoric \textit{so} must refer to something; likely not the post-insertion content)

\section{The concern of this work}

\begin{itemize}
    \item To capture the semantics of classificatory adjectives and taxonomic readings with an untyped semantics of \textcite{chierchia1984topicssyntaxsemantics}.
    \begin{itemize}
        \item The Neocarlsonian approach does not predict that relational nouns can be kind-referring, but they can.
    \end{itemize}
    \item To formalize the idea that WEK are referred to via proper names.
\end{itemize}

\section{Nominalized predicates}
\subsection{The HST* \parencite{cocchiarella1974fregeansemanticsrealist,chierchia1984topicssyntaxsemantics}}

An untyped lambda-calculus for natural language semantics.

\begin{itemize}
    \item Types are limited to
    \begin{itemize}
        \item individuals
        \item N-ary predicates over \textbf{entities only}
        \item functors (non-\textit{t}-ending)
    \end{itemize}
    \item All predicates have an individual counterpart.
    \begin{itemize}
        \item Predicates are turned individual by the Down-operator $^\cap $.
        \item Individuals are turned predicate by the Up-operator $^\cup $.
    \end{itemize}
    \item This effectively allows predicates to range over predicates.
    \begin{itemize}
        \item Including ranging over expressions of different types.
        \item Including self-predication and much more.
    \end{itemize}
\end{itemize}

\emph{Note.} $^\cup $ and $^\cap $ are not type-shifters or semantics of null expressions. They are composition principles and come for free.

\subsection{The Neocarlsonian semantics \parencite{chierchia1998referencekindslanguages}}

Kinds are functions from worlds to totalities of individuals that belong to that kind.

Linkean semantics for pluralities \parencite{link1983logicalanalysisplurals}: a join semilattice $\langle E,\oplus\rangle$ where elements above are mereological sums of elements below.
\begin{itemize}
    \item Part-of relation $\le$: e.g. $a\oplus b \le a\oplus b\oplus c$.
    \item A kind is the uppermost element in the lattice.
\end{itemize}

Rethinking \textcite{chierchia1984topicssyntaxsemantics}'s \textsc{down} and \textsc{up} operators as type-shifters.
\pex<>
    \a<down> $^\cap P = \lambda s\,\iota x\in D_k.\,P_s(x)$
    \a<up> $^\cup d = \lambda x.\,x\le d_s$
\xe

\textbf{Note.} It follows that only one-place predicates can correspond to kinds. This makes incorrect predictions (section \ref{rel}). Further employing the HST*: kinds are nominalized predicates with no connection to the content of the predicate.

\section{Classificatory adjectives}

\pex<cla>
    \a<a> technical architect
    \a<b> pulmonary disease
    \a<c> brown bear
\xe

Tightly related to their nouns.

\begin{itemize}
    \item Non-compositional semantics.
    \item Structurally lowest --- always linearily adjacent to the noun. In Lithuanian, unseparable by possessors \parencite{rutkowski2006classifyingadjectivesnoun}.    
\end{itemize}

\pex<>
    \a \begingl
        \gla žalia Reginos suknelė\trailingcitation{attributive}//
        \glb green Regina-GEN dress//
        \glft `Regina’s green dress’//
    \endgl
    \a \begingl
        \gla Reginos žalioji arbata\trailingcitation{classificatory}//
        \glb Regina-GEN green tea//
        \glft `Regina’s green tea’\trailingcitation{Lithuanian}//
    \endgl
\xe

\begin{itemize}
    \item Available, but limited in predicative position. Require a compatible noun in the subject.
    \begin{itemize}
        \item[$\implies$] not compounds.
    \end{itemize}
\end{itemize}

\pex<>
    \a<> This architect is technical.
    \a \{context: This guy is an architect.\}\\\ljudge*This guy is technical.
\xe

\emph{Note.} There can be more than one classificatory adjective.

\pex<>
    \a Scandinavian red fox
    \a This red fox is Scandinavian.
    \a \ljudge*This fox is Scandinavian red.
\xe


\subsection{Definite suffixes}

Some languages (Serbian \parencite{rutkowski2005classificationprojectionpolish}, Lithuanian \parencite{rutkowski2006classifyingadjectivesnoun,holvoet2012semanticmapdefinite}, Latvian \parencite{holvoet2012semanticmapdefinite}) mark noun phrase definiteness on adjectives.

\pex<lvnp>
\a<bare> \begingl
        \gla skudrlācis//
        \glb anteater//
        \glft `a/the anteater'//
    \endgl
    \a<attr> \begingl
        \gla skaist-s skudrlācis//
        \glb beautiful-\Nom{} anteater//
        \glft `a/*the beautiful anteater'//
    \endgl
    \a<def> \begingl
        \gla skaist-\emph{ai}-s skudrlācis//
        \glb beautiful-\Def-\Nom{} anteater//
        \glft `*a/the beautiful anteater'\trailingcitation{Latvian}//
    \endgl
\xe
        
The same marker is required on classificatory adjectives, without implying definiteness.
        
\ex
    \begingl
        \gla liel-\textbf{ai}-s skudrlācis//
        \glb big-\Def-\Nom{} anteater//
        \glft `a/the giant anteater (\textit{Myrmecophaga tridactyla})'//
    \endgl
\xe
        
\textbf{The idea here}: classificatory adjectives range over subkinds of the noun (or the noun with classificatory adjectives) \parencite[cf.][]{mcnally2004relationaladjectivesproperties}.

\section{Kinds and proper names}\label{pn}

WEK can be treated as proper names \parencite{carlson1977referencekindsenglish,heyer1985genericdescriptionsdefault,krifka1995genericityintroduction}

\pex<>
    \a<> \textit{Google} is so called because the creators dreamed of parsing a googol of pages.
    \a<> \ljudge* My neighbour is so called because he is the only living soul for miles.
    \a<> The anteater is so called because it eats ants.
    \a<> The giant anteater is so called because it can be more than two meters long.
\xe

Conceptual similarity: kinds are rigid designators \parencite{krifka1995genericityintroduction}.

Predicative position availability for classificatory adjectives is symmetrical to that of \textit{called}.

\pex<>
\a<> Such architects are called technical.
\a<> \ljudge*Such guys are called technical.
\xe

\subsection{What are proper names?}

Metalinguistic predicates that hold of all entities that bear the corresponding name \textcite[a.m.o.]{burge1973referencepropernames}. 

\pex<>
    \a<> I saw a Zachary today.
    \a<> There are many Zacharies here.
    \a<> The Zachary I told you about is following me.
\xe

Names in their standard usage (\textit{Zachary is crazy}) have a null determiner / are \textsc{iota} type-shifted: the unique most salient \textit{Zachary} \parencite{elbourne2005situationsindividuals}.

\section{The proposal}

Both nouns and classificatory adjectives are proper names.

\begin{itemize}
    \item One-place metalinguistic predicates over kinds.
    \begin{itemize}
        \item True for any kind that bears the name.
    \end{itemize}
    \item A \textsc{iota} is applied on every node in an NP's ext. projection to derive the unique kind.
    \item A predicate is true for individuals as well as its subkinds.
    \begin{itemize}
        \item For any well-established predicates $p$ and $q$, if $\forall x[p(x)\implies q(x)]$, then $q(^\cap p)$
    \end{itemize}
\end{itemize}

\pex<classem>
    \a<n> $\llbracket \text{anteater} \rrbracket = \lambda k.\,\textsc{called}(\text{anteater})(k)$
    \a<i> $\textsc{iota}\llbracket \text{anteater} \rrbracket = \iota k.\,\textsc{called}(\text{anteater})(k) = \textsc{anteater}$
    \a<up> $^\cup \textsc{iota}\llbracket \text{anteater} \rrbracket = \lambda x.\,\textsc{anteater}(x)$
    \a<a> $\llbracket \text{giant} \rrbracket  = \lambda k.\,\textsc{called}(\text{giant})(k)$
    \a<pm> $\llbracket \text{giant anteater} \rrbracket = \lambda x.\,\textsc{called}(\text{giant})(k) \land \textsc{anteater}(x)$
\xe

In simpler terms,
\begin{itemize}
    \item the \textsc{anteater} is the kind that is called "anteater".
    \item the \textsc{giant anteater} is the kind of anteater that is called "giant".
\end{itemize}

\textit{-ai-} is an opaque definiteness marker, with the semantics of \textsc{iota}. It naturally occurs on classificatory adjectives as well.

\pex<>
    \a<> $\llbracket \text{-ai-} \rrbracket (\llbracket  \text{liel- skudrlāci-}\rrbracket ) =  \iota k.\,\textsc{called}(\text{giant})(k) \land \textsc{anteater}(x)$
\xe

\section{Taxonomic NPs}

Determiners can range over subkinds. Singular \textit{the} returns the kind itself.

\pex<clasdet>
    \a<the> The anteater inhabits South America.
    \a<some> Some anteaters are extinct.
    \a<thepl> The anteaters are divided into two families.
\xe

Earlier: kinds are true of their subkinds.

\begin{itemize}
    \item It follows that determiners can range over subkinds.
\end{itemize}

\ex<>
    $\exists k.\,\textsc{anteater}(k) \land \textsc{extinct}(k)$
\xe

\begin{itemize}
    \item Kinds are also true of themselves.
    \item The totality of \textsc{anteater} subkinds is extensionally equal to \textsc{anteater}.
    \item The \textsc{anteater} is singular.
    \begin{itemize}
        \item[$\implies$] To refer to the whole kind, singular \textit{the} is used.
    \end{itemize}
\end{itemize}

\section{Additional: relational nouns against the Neocarlsonian semantics}\label{rel}

If we accept the analysis above, we assume that classificatory adjectives and nouns range over kinds upon combining.

Relational nouns (type $\langle e,t\rangle$) combine with classificatory adjectives: \textit{older brother}, \textit{personal assistant}, etc.

\ex<rel>
    \begingl
        \glpreamble \{Context: Anna has two older brothers.\}//
        \gla Anna-s vec-āk-\textbf{ai}-s brālis iegūva Nobela premiju//
        \glb Anna-\Gen{} old-\Comp-\Def-\Nom{} brother received Nobel's prize//
        \glft `An older brother of Anna has received Nobel's prize.'//
    \endgl
\xe

\begin{itemize}
    \item[$\implies$] Relational nouns denote kinds.
\end{itemize}

Neocarlsonian semantics cannot deal with it.
\begin{itemize}
    \item There is no totality of \textsc{brothers} to be the reference of the kind.
    \item The result of applying \textsc{up} is always one-place, predicts $\llbracket \text{older brother}\rrbracket$ to be one-place.
\end{itemize}

\section{Summary}

\begin{itemize}
    \item Well-established kinds --- nouns, classificatory adjectives, (compounds\dots) --- are underlyingly proper names.
    \item Classificatory adjectives range over subkinds.
    \item The kind-level singular \textit{the} explained in terms of self-predication.
    \item Neocarlsonian semantics does not capture the whole picture.
    \item Still much work to do.
    \begin{itemize}
        \item The external semantics of kind-referring NPs is yet to be developed.
        \item Everything is murky with ad-hoc kinds.
    \end{itemize}
\end{itemize}

\printbibliography

\end{document}