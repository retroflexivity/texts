\documentclass[a4paper, 12pt]{article}
\usepackage[left=2.5cm,
            right=2.5cm,
            top=2.5cm,
            bottom=2.5cm,
            bindingoffset=0cm]{geometry}
% \usepackage{array}
% \usepackage[indent=0pt]{parskip}
\usepackage{hyphsubst}
\usepackage{setspace}
\onehalfspacing

% \usepackage{float}
% \usepackage{graphicx}
% \graphicspath{ {./images/} }
% \usepackage{subfig}
% \usepackage{enumerate}
\usepackage[normalem]{ulem} % underlining
% \usepackage{booktabs} % tables
% \PassOptionsToPackage{table}{xcolor}% coloring tables

% LANGUAGE + FONT
		    
\usepackage[english, russian]{babel}

\usepackage{natbib}
\bibpunct[: ]{[}{]}{;}{a}{}{,}
\bibliographystyle{rusnat}

\usepackage{fontspec}  
\setmainfont{Gentium Plus}

% DRAWING

\usepackage{tikz}
\usepackage{tikz-qtree}
\usetikzlibrary{shapes.geometric}
\usetikzlibrary{trees,arrows}
\usetikzlibrary{positioning}

% LINGUISTICS 

%\usepackage{gb4e}
\usepackage{expex}
\lingset{aboveglftskip=0ex, belowglpreambleskip=0ex, belowexskip=1ex, aboveexskip=1ex, interpartskip=1ex}

\gathertags
\usepackage[glossaries]{leipzig}
\newleipzig{cmpr}{cmpr}{Comparative}
\newleipzig{sprl}{sprl}{Superlative}
\newleipzig{indef}{indef}{Indefinite}
% \makeglossaries

\usepackage{stmaryrd}

% MATH
\usepackage{amssymb}

\begin{document}
% \begin{sloppypar}

% -------------------------- текстиктекстиктекстик -----------------------------

\textbf{Определенные неиндивиды в латышском языке}

\textit{Тезисы на секцию «нелексикалистские подходы к языковым явлениям»}\\\\
В латышском языке категория определенности именной группы маркируется на флексии прилагательных: они могут иметь «определенную» и «неопределенную» форму (\nextx).

\pex<base>
    \a \begingl
        \gla balt-s krekls//
        \glb белый-\M{}.\Indef{} рубашка//
        \glft 'a white shirt'//
    \endgl
    \a \begingl
        \gla balt-ais krekls//
        \glb белый-\M{}.\Def{} рубашка//
        \glft 'the white shirt'//
    \endgl
\xe

Тем не менее, этот маркер не влечет неопределенности именной группы. Определенное множество прилагательных требует определенной формы даже в неопределенных контекстах. Среди них --- классифицирующие, или родовые (kind-referring), прилагательные, то есть такие, которые вместе с существительным выражают «хорошо устоявшиеся классы» \citep{carlson1977, trugman2005} (\nextx).

\pex<kind>
    \a<nkind> \begingl
        \gla balt-s lācis//
        \glb белый-\M.\Indef{} медведь//
        \glft 'белый медведь (медведь-альбинос)'//
    \endgl 
    \a<kind> \begingl
        \gla balt-ais lācis//
        \glb белый-\M.\Def{} медведь//
        \glft 'белый (полярный) медведь'//
    \endgl
\xe

В (\getfullref{kind.nkind}) \textit{balts} --- обычное интерсективное прилагательное, выражающее цвет индивида. В (\getfullref{kind.kind}) \textit{baltais} указывает на вид медведя, и словосочетание некомпозиционально: семантический вклад \textit{белый} здесь уникален и отличается, например, от такового в подобном же словосочетании \textit{белый чай}.

При этом ИГ типа \textit{baltais lācis} может реферировать и к уже вызванному в дискурсе медведю, и к новому. Что еще более необычно --- такие ИГ сохраняют маркер определенности даже под кванторами:

\ex<qu> \begingl
        \gla katr-s politisk-ais spēks pārstāv savu notikumu interpretāciju//
        \glb каждый-\M{} политический-\M.\Def{} сила представляет свою событий интерпретацию//
        \glft 'Каждая политическая сила представляет свою интерпретацию событий'//
    \endgl
\xe

В (\getref{qu}) ИГ в единственном числе, поэтому нельзя прибегнуть к объяснению через частичный квантор в духе \citep{matt2023}. Мы можем прийти к выводу, что такие именные группы действительно семантически не являются определенными.

Однако что, если не собственно определенность, в таком случае лицензирует маркер определенности? \citep{rutpro2006} предлагают для литовского синтаксический анализ, где маркер определенности --- рефлекс передвижения существительного в специальную функциональную проекцию ClassP. Но такой анализ упускает важное семантическое обобщение: вышеупомянутое требование, чтобы классифицирующий предикат выражал устоявшийся класс, очень похоже на требование возбужденности индивида в дискурсе для определенности в обычных контекстах.

Действительно, возможность определенного окончания прямо коррелирует с существованием концепта в общем дискурсе. Сравните (\nextx).

\pex<disc>
    \a<cmn> \begingl
        \gla šodien uz ielas atradu elektrisk-o (\judge{*}-u) tējkann-u//
        \glb сегодня на улице нашел электрический-\M.\Def.\Acc{} (-\Indef) чайник-\Acc{}//
        \glft 'Сегодня я нашел на улице электрический чайник'//
    \endgl
    \a<uncmn> \begingl
    \gla šodien uz ielas atradu elektrisk-u (\judge{*}-o) zirnekl-i//
    \glb сегодня на улице нашел электрический-\M.\Indef.\Acc{} (-\Def) паук-\Acc{}//
    \glft 'Сегодня я нашел на улице электрического паука'//
\endgl
\xe

Электрические чайники (\getfullref{disc.cmn}) --- известные индивиды в широком постиндустриальном дискурсе. Электрические пауки же встречаются реже, потому классифицирующее прочтение для такого предиката недоступно. В тех же терминах \textit{baltais krekls} в (\getfullref{ldisc.cmn}) --- известный в локальном дискурсе индивид, а \textit{balts krekls} (\getfullref{ldisc.uncmn}) --- неизвестный.

\pex<ldisc>
    \a<cmn> \begingl
        \gla rīt atkal uzvilkšu balt-o krekl-u//
        \glb завтра опять надену белый-\M.\Def.\Acc{} рубашка-\Acc{}//
        \glft 'Завтра опять надену (ту же) белую рубашку'//
    \endgl
    \a<uncmn> \begingl
        \gla rīt nopirkšu balt-u krekl-u//
        \glb завтра куплю белый-\M.\Indef.\Acc{} рубашка-\Acc{}//
        \glft 'Завтра куплю белую рубашку'//
    \endgl
\xe

С другой стороны, названия сортов всегда требуют определенного окончания (\nextx). Этим они похожи на имена собственные, которые ведут себя как определенные индивиды (не позволяют прочтений de dicto и пр.).

\ex<proper>
    \begingl
        \gla šo jauno arbūza šķiru es nosaukšu par balt-o arbūz-u//
        \glb этот новый арбуза сорт я назову о белый-\M.\Def.\Acc{} арбуз-\Acc{}//
        \glft 'Этот новый сорт арбуза я назову белым арбузом'//
    \endgl
\xe

Мы утверждаем, что определенное окончание способно маркировать определенность как индивида, так и предиката, причем ни одно не влечет другого. Описать это можно, например, через динамический тип окончания; на докладе мы представим такой анализ, углубимся в связанные с ним проблемы и опишем его следствия для общей семантики референции.

\printglossaries

\bibliography{ref}

% \end{sloppypar}
\end{document}