\documentclass{beamer}
\usefonttheme{serif}
\usecolortheme[named=black]{structure}
\setbeamertemplate{footline}[frame number]{}
\setbeamertemplate{navigation symbols}{}

\usepackage[normalem]{ulem} % underlining

% LANGUAGE + FONT
		    
\usepackage[english]{babel}

\usepackage{natbib}
\bibpunct[: ]{[}{]}{;}{a}{}{,}
\bibliographystyle{rusnat}

\usepackage{fontspec}  
\setmainfont{Gentium Plus}

% DRAWING

\usepackage{tikz}
\usepackage{tikz-qtree}
\usetikzlibrary{shapes.geometric}
\usetikzlibrary{trees,arrows}
\usetikzlibrary{positioning}

% LINGUISTICS 

%\usepackage{gb4e}
\usepackage{expex}
\lingset{aboveglftskip=0ex, belowglpreambleskip=0ex, belowexskip=1ex, aboveexskip=1ex, interpartskip=1ex}

\gathertags
\usepackage[glossaries]{leipzig}
\newleipzig{freq}{freq}{Frequentative}
\newleipzig{add}{add}{Additive particle}
\newleipzig{nfin}{nfin}{Non-Finite}
\newleipzig{npst}{npst}{Non-Past}
\newleipzig{int}{int}{Intensifier}
\newleipzig{pssg}{poss.2sg}{Discoursive non-possessive}
% \makeglossaries

\usepackage{stmaryrd}

\usepackage[linguistics]{forest}

% MATH
\usepackage{amssymb}

\title{IS-driven idiosyncrasies of the Northern Khanty SELF-intensifier}
\author{Arkady Shaldov}
\institute{HSE Moscow}
\date{TFS Osnabrück, 20.09.23}

\begin{document}

% -------------------------- текстиктекстиктекстик -----------------------------

\begin{frame}
    \titlepage
\end{frame}

\begin{frame}
    \frametitle{TOC}

    \tableofcontents

\end{frame}

\section{Intro}
\subsection{On Khanty}

\begin{frame}
    \frametitle{Northern Khanty, Kazym dialect}
    
    \begin{itemize}
        \item Ob-Ugric < Uralic
        \item Khanty-Mansi Autonomous Okrug, Middle Russia
        \item Spoken by some 10,000 people (mostly elderly) --- Threatened
        \item High dialect variation
    \end{itemize}
    
    ~\\~\\ Data gathered via direct elicitation (2022--2023, Kazym village)
    
\end{frame}

\subsection{On intensifiers}

\begin{frame}
    \frametitle{Intro: intensifiers}

    Focus expressions that contrast the individual with others related via a salient relation\\~\\

    Adnominal
    \pex
        \a Mr. Jones’ dog is indeed more sophisticated than Jones \textbf{himself}. (Jones vs Jones’ dog)
        \a Even the presenter \textbf{herself} didn’t believe in the hypothesis. (the presenter vs. the presenter’s opponents)
    \xe

    \only<1>{Adverbial (left out today)
    \ex
        The old woman crossed the street \textbf{herself}. (without help)
    \xe}
    
    \pause

    ~\\~\\Eckard's analysis [Eckardt 2001]:

    \begin{itemize}
        \item an intensifier is an identity function $\lambda x \in D_e. ID(x)$
        \item for it to be meaningful, it must be focused
        \item when it is focused, focus alternatives are relation functions (dog-of, opponent-of etc.)
    \end{itemize}


\end{frame}

\begin{frame}
    \frametitle{Lyutikova's [2001] hierarchy}

    Lyutikova proposes there are several readings of intensifiers (naming mine for consistency):

    \begin{itemize}
        \item contrastive: \textit{It was not John's brother but John himself.}
        \item additive: \textit{John himself agreed with the critics.}
        \item scalar: \textit{The King himself ordered it.}
        \item contrastively topicalized: \textit{John's wife left, John himself stayed.}
    \end{itemize}

    ~\\~\\There is a typological availability hierarchy of the readings:
    \ex additive, contrastive >> contrastive-topical, scalar \xe
\end{frame}

\section{Khanty data}
\subsection{Available readings}

\begin{frame}
    \frametitle{NKh adnominal intensifier}

    \textit{λʉw} (also 3SG personal pronoun; along w/ \textit{λiw} 3PL, \textit{λin} 3DU)\\~\\

    The contrastive and additive readings are available.
    
    \ex<base1> \begingl
        \glpreamble \{Is it Vasya's brother laying there in the snow?\}//
        \gla ăntɵ śit waśaj-en λʉw u-λ//
        \glb no this Vasya-\Pssg{} \Int{} lay-\Npst{}//
        \glft 'No, it's Vasya himself laying.'//
    \endgl    
    \xe
    \ex<base2> \begingl
        \glpreamble \{Is it true that Pasha's wife doesn't want to get a dog?\}//
        \gla pašaj-en λʉw ănt λăŋχa-λ amp tăj-ti//
        \glb Pasha-\Pssg{} \Int{} \Neg{} want-\Npst{} dog own-\Nfin.\Npst{}//
        \glft 'Pasha himself doesn't want to get a dog.'//
    \endgl    
    \xe
    
\end{frame}    


\begin{frame}
    \frametitle{\textit{λʉw} must have its \textit{intensificatum} topical}
    
    In the scalar reading, the individual is new to discourse and focused (this is likely what creates the scale).\\~\\
    
    \onslide<2>{NKh does not allow it}
    
    \ex<foc>\begingl
    \gla\ljudge{*}president λʉw waśaj-en sawot kɵšaj-a oməs-s-əλλe//
    \glb president \Int{} Vasya-\Pssg{} factory head-\Dat{} put-\Pst{}-\Tsg{}>\Sg{}//
    \glft 'The president himself appointed Vasya as the head of the factory.'//
    \endgl
    \xe
    
\end{frame}

\begin{frame}
    \frametitle{\textit{λʉw} must not be contrastively topicalized}
    
    \textit{λʉw} is only marginally available in contrastively topicalized reading.
    
    \ex<ct>
    \begingl
    \glpreamble \{Petya's wife went to the city\}//
    \gla \ljudge{*}petˊaj-en λʉw juλəŋ χaś-əs//
    \glb Petya-\Pssg{} \Int{} at\_home stay-\Pst{} //
    \glft 'Petya himself stayed at home.'//
    \endgl
    \xe
    
\end{frame}

\subsection{Syntactic positions}

\begin{frame}
    \frametitle{Syntactic positions}

    \textit{λʉw} is an \textbf{adnominal} intensifier\\~\\
    
    It would be expected an intensifier phrase is available in any syntactic position\\~\\
    
    \pause
    
    \onslide{But it's not the case}
    
    \begin{itemize}
        \item \textit{λʉw} can only be a subject of a phrase
    \end{itemize}    

\end{frame}    

\begin{frame}
    \frametitle{\textit{λʉw} can only be a subject of a phrase}

    \only<1>{
        \textit{λʉw} can be a subject of a clause, irrespective of the case it bears

        \ex<subj>
            \begingl
                \gla kašen rɵpatnik λʉw joχtə-s pa ime-λ tɵ-s//
                \glb every worker \Int{} come-\Pst{} and wife-\Poss.\Tsg{} bring-\Pst{}//
                \glft 'Every worker came himself and brought his wife.'//
            \endgl    
        \xe    

        \ex<dsubj>
            \begingl
                \gla mašaj-en λʉw-eλa śit ăn mos-λ, λʉw puχ-əλ-a mos-λ//
                \glb Masha \Int-\Dat{} this \Neg{} be\_needed-\Npst{} (s)he son-\Poss.\Tsg{} be\_needed-\Npst{}//
                \glft 'Masha herself doesn't need it, it's Masha's son who needs it.'//
            \endgl    
        \xe    
    }    

    \only<2>{
        \textit{λʉw} can be a possessor, i. e. a subject of a DP.

        \ex<poss> \begingl
            \glpreamble \{Is it Misha's father's parka?\}//
            \gla ăntɵ śit mišaj-en \textbf{λʉw} molśe-λ//
            \glb \Neg{} \Dem{} Misha-\Poss{}.\Ssg{} \Int{} parka-\Poss{}.\Tsg{}//
            \glft 'No, it's Misha's own parka.'//
        \endgl    
        \xe

    }    

    \only<3>{
        \textit{λʉw} cannot be an object

        \ex<do> \begingl
            \gla \ljudge*ma waśaj-en λʉw-ti ask-s-ɛm//
            \glb I Vasya-\Poss{}.\Ssg{} \Int-\Acc{} call-\Pst{}-\Fsg{}>\Sg{}//
            \glft ‘I asked Vasya himself \{but he sent his son instead\}.'//
            \endgl
        \xe    

        \ex<io>\begingl
            \glpreamble \{Did you tell it Pasha's wife?\}//
            \gla \ljudge*ma pašaj-en λʉw-eλa iśi śit oλəŋ-ən lup-s-əm//
            \glb I Pasha-\Pssg{}-\Dat{} \Int{}-\Dat{} \Add{} this about-\Loc{} tell-\Pst-\Fsg{}//
            \glft  'I told it Pasha himself, too.'//
        \endgl    
        \xe
    }    

    \only<4>{
        The subject of passive included

        \ex<pass>
            \begingl
                \glpreamble \{Why is Vasya's son going?\}//
                \gla waśaj-en λʉw mojaŋa woχ-s-a, puχ-əλ χɵn//
                \glb Vasya-\Pssg{} \Int{} to\_visit call-\Pst-\Pass{} son-\Poss.\Tsg{} \Neg{}//
                \glft 'Vasya himself was invited, not his son.'//
            \endgl    
        \xe    
    }    

\end{frame}    

\section{Analysis}

\begin{frame}
    \frametitle{Summarizing}

    \begin{itemize}
        \item \textit{λʉw} is only available as a subject of a phrase
        \item \textit{λʉw} requires its argument (intensificatum) to be topical
        \item \textit{λʉw} is poor in contrastive topic
    \end{itemize}

    How to unite the idiosyncrasies?\\~\\

    \pause

    \textbf{The get-out requirement}. the intensificatum must move out of the intensifier DP

    \begin{forest}
        for tree={s sep=30pt}
        [XP
            [DP [mašajen, roof, name=to]]
            [DP
               [DP, [mašajen, roof, name=from]]
               [DP [λʉw, roof]]
            ]
        ]
        \draw[->] (from) to[out=south west, in=south] (to);
    \end{forest}

\end{frame}

\begin{frame}
    \frametitle{The proposal}

    \textbf{The get-out requirement}. the intensificatum must (overtly) move out of the intensifier DP\\~\\

    The requirement can be satisfied via topic movement to TopP\\~\\

    \only<1>{
        There is overt topic movement in NKh indeed

        \ex
           \begingl
            \gla śi jupijn taməś woj ma ănt pa wantijλ-s-əm.//
            \glb \Dem{} after such beast I \Neg{} \Add{} see-\Pst{}-\Fsg{}//
            \glft 'Since then I haven't seen such a beast.'//
           \endgl 
        \xe
    }

    \pause 
    
    \begin{itemize}
        \item The movement is blocked when there is an argument higher
        \item The movement is impossible when the intensificatum is focused
        \item It is the whole intensifier DP that moves when it is contrastively topicalized
    \end{itemize}

\end{frame}

\begin{frame}
    \frametitle{The movement is blocked when there is an argument higher}

    \ex<do> \begingl
        \gla \ljudge*ma waśaj-en λʉw-ti ask-s-ɛm//
        \glb I Vasya-\Poss{}.\Ssg{} \Int-\Acc{} call-\Pst{}-\Fsg{}>\Sg{}//
        \glft ‘I asked Vasya himself \{but he sent his son instead\}.'//
        \endgl
    \xe

    \textit{ma} 'I' would move instead of \textit{waśajen}, as it is topical\\~\\

    It is still unclear if the intensifier would be allowed in double focus sentences
\end{frame}    


\begin{frame}
    \frametitle{Possessors}
    
    Where does the intensificatum move when \textit{λʉw} is a possessor?
    
    \ex<poss> \begingl
        \glpreamble \{I was searching for the house of Andrey's parents but\} //
        \gla ma antrej-en		λʉw	χot-əλ		wojət-s-ɛm//
        \glb I Andrey-\Pssg{}	\Int{}	house-\Poss.\Tsg{} find-\Pst-\Fsg>\Sg{}//
        \glft ‘I found the home of Andrey himself.’//
        \endgl
        \xe
        
    There is possessive agreement in NKh which seems to require a topical possessor\\~\\

    It is obligatory with \textit{λʉw}\\~\\
    
    We suggest it is because the intensificatum moves to Spec,PossP
\end{frame}

\begin{frame}
    \frametitle{The movement is impossible when the intensificatum is focused}
    
    \ex<foc>\begingl
    \gla\ljudge{*}president λʉw waśaj-en sawot kɵšaj-a oməs-s-əλλe//
    \glb president \Int{} Vasya-\Pssg{} factory head-\Dat{} put-\Pst{}-\Tsg{}>\Sg{}//
    \glft 'The president himself appointed Vasya as the head of the factory.'//
    \endgl
    \xe
    
    As \textit{president} is new and focused, it cannot be extracted
    
\end{frame}

\begin{frame}
    \frametitle{It is the whole intensifier DP that moves when it is contrastively topicalized}
    
    \ex<ct>
        \begingl
        \glpreamble \{Petya's wife went to the city\}//
            \gla \ljudge{*}petˊaj-en λʉw juλəŋ χaś-əs//
            \glb Petya-\Pssg{} \Int{} at\_home stay-\Pst{} //
            \glft 'Petya himself stayed at home.'//
    \endgl
    \xe
                        
    % \begin{forest}
    %     for tree={s sep=30pt}
    %     [TopP
    %         [DP
    %             [DP{[TOP]} [petjajen, roof, name=specint]]
    %             [DP [λʉw, roof]]
    %         ]
            % [DP
            %     [DP, [mašajen, roof, name=from]]
            %     [DP [λʉw, roof]]
            % ]
            % [TP
                % [DP{[TOP]} [\st{petjajen λʉw}, roof, name=subj]]
            
                % [VP [juλəŋ χaśəs, roof]]
        %     ]
        % ]
        % \draw[->] (from) to[out=south west, in=south] (to);
    % \end{forest}
\end{frame}

\begin{frame}
    \frametitle{The constraint might be phonological}

    Phonologically empty DPs don't obey the constraint\\~\\

    \begingl
        \gla mašaj-en		aŋki	uš-a		wɛr-s-əm,	∅	λʉw-ti		śit	ăntɵ//
        \glb Masha-\Poss.\Ssg{}	mother mind-\Dat{} make-\Pst-\Fsg{}	PRO \Int{}-\Acc{}	\Dem{}	\Neg{}//
        \glft 'Masha's mother, I recognized, but her(self), I did not.'//
    \endgl

    \pause 
    ~\\~\\ Such an anaphora would be unexpected, were \textit{λʉw} a personal pronoun\\~\\

    It must be an intensifier with a PRO --- despite contrastively topicalized on left periphery and accusative.

\end{frame}

\section{Conclusion}

\begin{frame}
    \frametitle{Concluding}

    \begin{itemize}
        \item We have seen that NKh intensifier obeys Lyutikova's hierarchy: additive, contrastive >> 
        *contrastive-topical, *scalar
        \item Together with the subjecthood requirement, they result in three constraints to be explained
        \pause
        \item The explanation: a constraint that the intensificatum be extracted from the DP
        \pause
        \item The focused-intensifier-topical-intensificatum requirement might be the key to solving Lyutikova's hierarchy
        
        
    \end{itemize}

\end{frame}

\end{document}