\documentclass[
    9pt,
]{beamer}
\usecolortheme[named=black]{structure}
\setbeamertemplate{footline}[frame number]{}
\setbeamertemplate{navigation symbols}{}
\setbeamertemplate{itemize items}{$\cdot$}

\setlength{\parskip}{0.5em}

\usepackage[russian,english]{babel}


% TYPOGRAPHY
\usepackage{amsmath}
\usepackage{stmaryrd}
% \usepackage{mathspec}
\usepackage{fontspec}
\usepackage{unicode-math}

\setmainfont{Libertinus Serif}
\setsansfont{Libertinus Sans}
\setmathfont{Libertinus Math}

\usepackage[normalem]{ulem}

\usepackage[dvipsnames]{xcolor}
\let\emph\relax
\DeclareTextFontCommand{\emph}{\color{BrickRed}}


% LINGUISTICS 
\usepackage{expex}
\lingset{aboveglftskip=-1ex, belowglpreambleskip=0ex, belowexskip=1ex, aboveexskip=1ex, interpartskip=0ex}
\gathertags
\let\expexgla\gla % dumb unicode-math conflict
\AtBeginDocument{\let\gla\expexgla}

\usepackage[glossaries]{leipzig}
\newleipzig{cmpr}{cmpr}{Comparative}
\newleipzig{sprl}{sprl}{Superlative}
\newleipzig{indef}{indef}{Indefinite}
% \newleipzig{def}{def}{Definite}


% SUBSTITUTION
\usepackage{hyphsubst}
\usepackage{csquotes}
\MakeOuterQuote{"}


% BIBLIOGRAPHY		    
\usepackage[style=authoryear,url=false,doi=false,isbn=false,eprint=false,date=year]{biblatex}
\addbibresource{../../ref.bib}


% DRAWING
% \usepackage{tikz}
% \usetikzlibrary{shapes.geometric}
% \usetikzlibrary{trees,arrows}
% \usetikzlibrary{positioning}
\usepackage[linguistics]{forest}


%ALIASES
\ExplSyntaxOn

\NewDocumentCommand{\splist}{ s m m m} {
    #3\mspace{2mu}
    \clist_set:Nn \l_tmpa_clist { #2 }
    \clist_use:Nn \l_tmpa_clist {,\mspace{3mu plus 1mu minus 1mu} \IfBooleanT{#1}{\allowbreak}}
    \mspace{2mu}#4
}

\NewDocumentCommand{\opair}{ s m }{
    \IfBooleanTF{#1}{ \splist*{#2}{\langle}{\rangle} }{ \splist{#2}{\langle}{\rangle} }
}

\NewDocumentCommand{\set}{ s m } {
    \IfBooleanTF{#1}{ \splist*{#2}{\{}{\}} }{ \splist{#2}{\{}{\}} }
}
\ExplSyntaxOff

\newcommand{\denote}[1]{\llbracket #1 \rrbracket}

\newcommand{\up}{\ensuremath{{}^\cup}}
\newcommand{\down}{\ensuremath{{}^\cap}}

%META
\title{Видовая референция}
\author{Аркадий Шалдов}
\date{27.11.2024, МЛ ЛогЛинФФ}

\begin{document}
% -------------------------- текстиктекстиктекстик -----------------------------

\begin{frame}
    \titlepage
\end{frame}


\section{Введение}

\begin{frame}
    \frametitle{Введение}

    Именные группы могут реферерировать не только к индивидам, но и к таксономическим единицам — \emph{видам}.

    Такие прочтения возможны со специальными предикатами видового уровня (\getfullref{intr.kl}-\getref{intr.inv}), с прилагательными (\getfullref{intr.adj}) и с генерическими предикатами (\getfullref{intr.hab}).

    \pex<intr>
    \a<kl> \emph{The mammoth} is extinct.\\
            \emph{Мамонт} вымер.
    \a<inv> Babbage invented \emph{the computer}.\\
            Бэббидж изобрел \emph{компьютер}.
    \a<adj> \emph{The engineer} is a good profession.\\
            \emph{Инженер} — хорошая профессия.
    \a<hab> \emph{The anteater} eats ants.\\
            \emph{Муравьед} питается муравьями.
    \xe

\end{frame}

\begin{frame}
    \frametitle{Таксономическая иерархия}

    Кроме того, язык позволяет квантифицировать по подвидам.

    \pex<>
        \a<> Every anteater inhabits South America.\\
        Все муравьеды обитают в Южной Америке.
        \a<> Some anteaters are extinct.\\
        Некоторые муравьеды вымерли.
        \a<> An / one anteater's conservation status says "threatened".\\
        Охранный статус одного муравьеда — "угрожаемый".
        \a<> The anteaters are divided into two families.\\
        Муравьеды делятся на два семейства.
    \xe

    Также: квантификация по видам масс.

    \ex<>
        В ресторане подается три разных супа.
    \xe

    Таксономическая предикация очень напоминает обычную, индивидную.
\end{frame}

\begin{frame}
    \frametitle{Два типа отношений}

    Предикативное отношение между двумя именами может иметь две разных природы (contra \textcite{dayal2004numbermarkingdefiniteness})

    \begin{itemize}
        \item собственно предикативная
        \begin{itemize}
            \item между индивидом и видом: бульдог(Шарик), инженер(Михаил Петрович)
            \item между видом и видом вида: порода(бульдог), профессия(инженер)
        \end{itemize}
        \item таксономическая
        \begin{itemize}
            \item между подвидом и видом: собака(бульдог), суп(солянка)
            \item НЕ между индивидом и множеством индивидов: Могучая\_кучка(Кюи)?
        \end{itemize}
    \end{itemize}

    Последнее к тому же, видимо, ограничено биологическими таксонами и именами масс

    \pex<>
        \a<> Бульдог — хорошая собака.
        \a<> Солянка — хороший суп.
        \a<> \ljudge* Семантист — хороший лингвист.
    \xe

    \textcite[и мн. др.]{krifka1995genericityintroduction}: это ингерентное свойство имен
    \begin{itemize}
        \item $\denote{\text{муравьед}}_P = \lambda x.\; \text{муравьед}(x)$
        \item $\denote{\text{муравьед}}_T = \lambda p.\; p\subset \text{муравьед} \land p\in K$
    \end{itemize}

\end{frame}

\begin{frame}
    \frametitle{Устоявшиеся и адгоковые виды}

    \begin{itemize}
        \item Видовые предикаты сочетаются не только с ед. ч.
        \item Не в единственном числе они сочетаются как с таксономическими сущностями, так и с \textit{адгоковыми видами} \parencite{mendia2019referenceadhoc}
    \end{itemize}
    
    \ex<>
        Тигры с белой шерстью очень редки.
    \xe

    По адгоковым видам нельзя квантифицировать.
    
    \ex<>
        \ljudge\# В зоопарке два тигра: с белой шерстью и с рыжей.
    \xe

    Я предполагаю, что онтологически устоявшиеся виды входят в домен видов, а адгоковые не входят. Про них мы говорить не будем.

\end{frame}


\section{Формализации}

\begin{frame}
    \frametitle{Вид = предикат}

    Например, \parencite{cocchiarella1974fregeansemanticsrealist,chierchia1984topicssyntaxsemantics} — логика HST*
    \begin{itemize}
        \item Каждому предикату онтологически соответсвует индивид
        \item Функции принимают индивидов, в т. ч. «номинализованные» предикаты
    \end{itemize}

    Менее радикальный подход — частный полиморфизм
    \pex<>
        \a<> $\denote{the} = \lambda p_{\langle\sigma,\; \tau\rangle}.\; \lambda x_\sigma: |p|\le 1 \land p(x)$\trailingcitation{адапт. \parencite{coppock2015definitenessdeterminacy}}
        \a<> \dots
    \xe

\end{frame}

\begin{frame}
    \frametitle{Неокарлсоновская семантика \parencite{chierchia1998referencekindslanguages}}

    Попытка интегрировать виды в современную лингвистическую онтологию — семантику множественного числа \textcite{link1983logicalanalysisplurals}

    \begin{itemize}
        \item Домен индивидов — верхняя полурешетка $\opair{E, \oplus}$ по мереологической сумме
        \item Нижние элементы — единичные индивиды, соотв. именам в ед. ч., элементы над ними — множественные индивиды
    \end{itemize}

    \textcite{chierchia1998referencekindslanguages}:

    \begin{itemize}
        \item Виды — это интенсиональные максимальные индивиды (верхние грани полурешеток)
        \item Два тайпшифтера:
        \begin{itemize}
            \item $\down P = \lambda s .\; \iota x \in D_k.\; P_s(x)$ — превращает предикат в вид
            \item $\up d = \lambda x.\; x\le d_s$ — превращает вид в предикат
        \end{itemize}
    \end{itemize}

    \ex<>
        $\denote{\down{\text{собака}}}^s = d_1\oplus d_2\oplus \dots$
    \xe
\end{frame}


\section{Классифицирующие прилагательные}

\begin{frame}
    \frametitle{Классифицирующие прилагательные}

    Явление, тесно связанное с идеей (под)видов.

    \pex<rel>
        \a<be> белый медведь
        \a<> технический архитектор
        \a<> легочное заболевание
    \xe

    Очень тесно связаны с модифицируемыми ими именами
    \begin{itemize}
        \item семантически: некомпозициональны — вклад прилагательного определяется существительным
        \pex<>
            \a<> белый чай
            \a<> белый медведь
        \xe
        \item синтаксически: неразрываемы другими прилагательными
        \ex<>
            маленький серый гигантский муравьед / \#маленький гигантский серый муравьед
        \xe
    \end{itemize}

\end{frame}

\begin{frame}
    \frametitle{Классифицирующие прилагательные как модификаторы видов}

    \textcite{mcnally2004relationaladjectivesproperties}: как обычные прилагательные модифицируют индивидов, так классифицирующие прилагательные модифицируют виды.

    \begin{itemize}
        \item[$\implies$] В именной группе сперва происходит рестрикция на уровне видов, затем — на уровне индивидов.
    \end{itemize}

    \pex<>
        \a маленький серый гигантский муравьед\\
        = `маленький серый представитель вида гигантских муравьедов'
        \a пожилой технический архитектор\\
        = `пожилой представитель профессии технических архитекторов'
    \xe

\end{frame}

\begin{frame}
    \frametitle{Языки с определенностью на прилагательных}

    В некоторых языках (сербский \parencite{rutkowski2005classificationprojectionpolish}; литовский, латышский \parencite{holvoet2012semanticmapdefinite}) прилагательные могут иметь суффикс, прим. соответствующий англ. \textit{the}.
    
    Латышский:
    \pex<>
        \a<> \begingl
            \gla lācis//
            \glb медведь//
            \glft `a/the bear'//
        \endgl
        \a<> \begingl
            \gla liels lācis//
            \glb большой медведь//
            \glft `a big bear'//
        \endgl
        \a<> \begingl
            \gla liel-ai-s lācis//
            \glb большой-\Def{} медведь//
            \glft `the big bear'//
        \endgl
    \xe

\end{frame}

\begin{frame}
    \frametitle{Определенность классифицирующих прилагательных}

    При этом тот же самый маркер обязателен на классифицирующих прилагательных, даже если вся именная группа неопределенная

    \pex<>
        \a \begingl
            \gla balt-ai-s lācis//
            \glb белый-\Def{} медведь//
            \glft `a/the polar bear'//
        \endgl
        \a \begingl
            \gla \ljudge\#balts lācis//
            \glb белый медведь//
            \glft только `a white-colored bear'//
        \endgl
    \xe

    Интуиция: классифицирующие прилагательные с существительными выражают устоявшийся, т. е. определенный, вид $\implies$ определенность $\implies$ маркер определенности
    
\end{frame}

\begin{frame}
    \frametitle{Формализация интуиции}

    \pex<>
        \a $\denote{\text{-ai-}} = \denote{the} = \lambda P.\; \lambda x: |P|\le 1.\; P(x)$\trailingcitation{\parencite{coppock2015definitenessdeterminacy}}
        \a $\denote{\text{-ai-}}(\text{большой медведь}) = \lambda x_e: |\text{большой} \cap \text{медведь}_P| \le 1.\allowbreak \text{большой}(x) \land \text{медведь}_P(x)$\trailingcitation{определенность индивида}
        \a $\denote{\text{-ai-}}(\text{белый медведь}) = \lambda p_{\opair{e, t}}: |\text{белый} \cap \text{медведь}_T| \le 1.\allowbreak \text{белый}(p) \land \text{медведь}_T(p)$\trailingcitation{определенность вида}
    \xe

    \emph{Максимизируй пресуппозицию \parencite{heim1991artikelunddefinitheit}}: из двух эквивалентных предложений использовать то, которое имеет более сильную пресуппозицию (если она удовлетворяется).

    Далее — экзистенциальное / йота-замыкание \parencite{coppock2015definitenessdeterminacy} превращает предикат по видам в вид (предикат по индивидам)

    \ex<>
        $\textsc{iota}(\denote{\text{-ai-}}(\text{белый медведь})) = \lambda x_{e}.\; (\iota p.\;\text{белый}(p) \land \text{медведь}_T(p))_P(x) = \lambda x_e.\; \text{белый\_медведь}_P(x)$
    \xe

\end{frame}

\begin{frame}
    \frametitle{В неокарлсоновской семантике}

    Имена изначально — предикаты по видам, к которым применяется сначала \textsc{iota}\footnotemark, затем — $\up$ \parencite[ср.][]{sereikaite2017kindreferencedps}.

    \footnotetext{\textsc{iota} = \lambda p. \iota x.\; p(x) \parencite{partee1986nounphraseinterpretation}}

    \pex<>
    \a $\denote{\text{-ai-}} = \denote{the} = \lambda p.\; \lambda x: |p|\le 1.\; p(x)$\trailingcitation{\parencite{coppock2015definitenessdeterminacy}}
    \a $\denote{\text{-ai-}}(\text{большой медведь}) = \lambda x_e: |\text{большой} \cap \text{медведь}_P| \le 1.\allowbreak \text{большой}(x) \land x \le \text{медведь}_P$\trailingcitation{определенность индивида}
    \a<> $\denote{\text{-ai-}}(\text{белый медведь}) = \lambda k: |\text{белый} \cap \text{медведь}_T| \le 1.\allowbreak \text{белый}(p) \land \text{медведь}_T(p)$\trailingcitation{определенность вида}
    \a<> $\up \denote{\text{-ai-}}(\text{белый медведь}) = \lambda x_{e}.\; x \le (\iota p.\;\text{белый}(p) \land \text{медведь}_T(p))$
    \xe

\end{frame}

\begin{frame}
    \frametitle{Проблема: реляционные имена}

    Реляционные имена имеют арность 2, поэтому не соответствуют множеству индивидов.
    
    При этом реляционные имена также сочетаются с классфицирующими прилагательными

    \ex<>
        \begingl
            \gla vecāk-ai-s brālis//
            \glb старший-\Def{} брат//
            \glft `a/the older brother'//
        \endgl
    \xe

    В неокарлсоновском подходе
    \begin{itemize}
        \item вид — множество индивидов, но \textit{брат} не соответствует множеству индивидов.
        \item $\up$ с необходимостью возвращает одноместный предикат, но \textit{старший брат} — двуместный предикат.
    \end{itemize}

\end{frame}

\begin{frame}
    \frametitle{Суммируя}

    \begin{itemize}
        \item Классифицирующие прилагательные хочется анализировать как модификаторы видов
        \item Это хорошо объясняет данные из языков с определенностью на прилагательных
        \item Эти же данные показывают, что классифицирующие прилагательные сочетаются с реляционными именами
        \item Но неокарлсоновская семантика противоречит тому, что реляционные имена соответствуют видам
    \end{itemize}

\end{frame}


\section{К значению}

\begin{frame}
    \frametitle{К значению}

    Все еще вопрос: что же такое белый в $\lambda x_{e}.\; (\iota p.\;\text{белый}(p) \land \text{медведь}_T(p))_P(x)$?

    \parencite{krifka1995genericityintroduction}: виды как имена

    \textit{так-называемый}-тест:

    \pex<>
        \a<> Медведи так называются, потому что они хорошо ищут мед.
        \a<> \ljudge* Красивые медведи так называются, потому что у них большие черные глаза.
        \a<> Белые медведи так называются, потому что у них белая шерсть.
    \xe

    Сочетания из классифицирующего прилагательного и существительного (по крайней мере некоторые) — имена собственные, в отличие от сочетаний с аттрибутивными прилагательными.

\end{frame}

\begin{frame}
    \frametitle{К значению}

    При этом существительное в таком сочетании проваливает тест:

    \pex<>
        \a<> \ljudge{\textsuperscript{??}}Белые медведи так называются, потому что хорошо ищут мед.
        \a<> Белый медведь — вид медведя.
        \a<> \ljudge*Белый медведь — вид белого.
    \xe

    \emph{Белый медведь — вид медведя, который называется белым.}

\end{frame}

\begin{frame}
    \frametitle{Имена}

    Именной предикативизм \parencite{burge1973referencepropernames}: имя — предикат, утверждающий крещение.
    \pex<>
        \a<> Я Петя. = $N$("Петя")(я)
        \a<> У нас на курсе ни одной Светы. = $\neg\exists x\;(\text{у\_нас\_на\_курсе}(x)\ \land N(\text{"Света"})(x))$
        \a<> Я видел Бориса. = $\text{видел}(\text{я})(\iota x.\; N(\text{"Борис"})(x))$
    \xe

    С видами так же

    \pex<>
        \a<> $\denote{\text{медведь}} = \lambda x.\; N(\text{медведь})(x) $
        \a<> $\textsc{iota}(\denote{\text{медведь}}) = \iota x.\; N(\text{медведь})(x) = \lambda y.\; \text{медведь}(y)$
        \a<> $\denote{\text{белый}} = \lambda y.\; N(\text{белый})(y)$
        \a<> $\denote{\text{белый медведь}} = \lambda y.\; N(\text{белый})(y) \land \text{медведь}_T(y)$
        \a<> $\textsc{iota}\denote{\text{белый медведь}} = \iota y.\; N(\text{белый})(y) \land \text{медведь}_T(y) = \lambda w.\; \text{белый\_медведь}(w)$
    \xe

\end{frame}

\begin{frame}
    \frametitle{Заключение}

    \begin{itemize}
        \item Мы посмотрели на то, как можно смотреть на виды
        \item Неокарлсоновская семантика относится к видам как к максимальным индивидам
        \item Это сталкивается с проблемами, если учесть, что видам соответсвуют в т. ч. реляционные имена
        \item Мы также посмотрели, как можно смотреть на классифицирующие прилагательные
        \item Для некоторых из них кажется правдоподобным крипкеанский анализ
        \item Но не для всех
    \end{itemize}

\end{frame}

\end{document}
