\documentclass[a4paper, 12pt]{article}
\usepackage[left=2.5cm,
            right=2.5cm,
            top=2.5cm,
            bottom=2.5cm,
            bindingoffset=0cm]{geometry}
% \usepackage{array}
% \usepackage[indent=0pt]{parskip}
\usepackage{hyphsubst}
\usepackage{setspace}
\singlespacing

\usepackage{hyperref}
% \usepackage{float}
% \usepackage{graphicx}
% \graphicspath{ {./images/} }
% \usepackage{subfig}
% \usepackage{enumerate}
\usepackage[normalem]{ulem} % underlining
% \usepackage{booktabs} % tables
% \PassOptionsToPackage{table}{xcolor}% coloring tables

% LANGUAGE + FONT
		    
\usepackage[english, russian]{babel}

\usepackage{natbib}
\renewcommand{\bibsection}{~\\\textbf{References}}
\bibpunct[: ]{[}{]}{;}{a}{}{,}
\bibliographystyle{rusnat}

\usepackage{fontspec}  
\setmainfont{Gentium Plus}

% DRAWING

\usepackage{tikz}
\usepackage{tikz-qtree}
\usetikzlibrary{shapes.geometric}
    \usetikzlibrary{trees,arrows}
\usetikzlibrary{positioning}

% LINGUISTICS 

%\usepackage{gb4e}
\usepackage{expex}
\lingset{aboveglftskip=0ex, belowglpreambleskip=0ex, belowexskip=1ex, aboveexskip=1ex, interpartskip=0ex}

\gathertags
\usepackage[glossaries]{leipzig}
\newleipzig{cmpr}{cmpr}{Comparative}
\newleipzig{sprl}{sprl}{Superlative}
\newleipzig{indef}{indef}{Indefinite}
% \makeglossaries

\usepackage{stmaryrd}

% MATH
\usepackage{amssymb}

\begin{document}
\begin{sloppypar}

% -------------------------- текстиктекстиктекстик -----------------------------

\textbf{Baltic evidence to external aspect}\\~


Aspect in Latvian is often described in the same terms as Slavic aspect [2], as it also has
perfectivising prefixes (\nextx). These prefixes add telic interpretation to the verb as well as
normally change aspect to perfective. Some of them additionally induce lexical meaning
change; others, like \textit{no-} in (\getfullref{base.pfv})\footnote{This and other Russian examples are constructed as the authors are native speakers of Russian. Latvian
examples are elicited, unless stated otherwise.}, do not.

\pex<base>
    \a<ipfv> \begingl
        \gla Katru dienu sūtī-ju viņam vēstules//
        \glb every day send-\Pst{} him letters//
        \glft ‘I used to send him letters every day.’//
    \endgl
    \a<pfv> \begingl
        \gla Vakar no-sūtī-ju viņam vēstuli//
        \glb yesterday \Pfv-send-\Pst{} him letter//
        \glft ‘I sent him a letter yesterday.’//
    \endgl    
\xe

(\getfullref{base.ipfv}) is understood as a habitual action in the past, whereas (\getfullref{base.pfv}) is an episodic action. Same opposition applies in Russian (\nextx).

\pex<>
    \a<ipfv> \begingl
        \gla Každy den’ ja sla-l jemu pis’ma//
        \glb every day I send-\Pst{} him letters//
        \glft ‘I used to send him letters every day.’//
    \endgl
    \a \begingl
        \gla Včera ja po-sla-l jemu pis’mo//
        \glb yesterday I \Pfv-send-\Pst{} him letter//
        \glft ‘I sent him a letter yesterday.’//
    \endgl
\xe

In Slavic languages all imperfective contexts require imperfective morphology [1]. When carrying present morphology, verbs with
perfective prefixes refer to future. Conversely, Latvian allows the so-called perfective present (PP), where verbs with perfective prefixes are used in typical present contexts [2]. Although it is unavailable as progressive (\textit{I am sending him a letter right now} in Latvian is expressed with an imperfective verb), perfective is felicitous in habituals and praesens historicum (\nextx).

\pex<pp>
    \a<hab> \begingl
        \gla Varētu teikt, ka \textbf{ne-ēdu} kūciņas, bet tad es melotu — gadā vienu, varbūt divas kūciņas \textbf{ap-ēdu}.//
        \glb can.\Cond{} say that not-eat.\Prs{} cakes but then I lie.\Cond{} {} year one maybe two cakes \Pfv-eat.\Prs{}//
        \glft ‘I could say that I don’t eat cakes, but then I would be lying – I eat one, maybe two cakes per year.’ \trailingcitation{[2]}//
    \endgl

    \a<ph> \begingl 
        \gla dzejnieks pārsvītro visus trīs “Tu” un vietā \textbf{uz-raksta} “Es”//
        \glb poet cross.out.\Prs{} all three “You” and instead \Pfv-write.\Prs{} “I”//
        \glft ‘[In the fourth part of the action ...] the poet crosses out all three “You” and writes “I” instead.’\trailingcitation{[3]}//
    \endgl
\xe

Overall, PP is available when the culmination of the event is asserted. In progressive, on the contrary, it is outside of the evaluation time, so prefixless form is used.

Still, Slavic languages require imperfective in such contexts. When telic interpretation is needed
in present, Russian employs imperfectivising affixes like -yva- and -a-. Compare (\getref{pp.hab}) with the semantically identical Russian (\nextx).

\ex<rus>
    \begingl
        \gla {na samom dele} ja \textbf{s-jed-a-ju} odin-dva v god.//
        \glb {in fact} I [\Pfv-eat]-\Ipfv-\Prs{} one-two per year//
        \glft ‘[One could say that I don’t eat cakes, but] in fact I eat one-two cakes per year.’ //
    \endgl
\xe

[2] proposes that Latvian verbs are bi-aspectual, but that is problematic [1]. In [1] authors
claim that the difference between the Baltic and Slavic system is the degree of
grammaticalisation. More precisely, "[in Slavic] the perfective present has basically become a
perfective future", however in Baltic languages it is not a strict requirement.
We are to employ S. Tatevosov’s analysis [4] of Russian, which suggests that aspect is
external to the verb. The difference between verb-internal and verb-external aspect is (\nextx).

\pex<>
    \a<> [CP … [F\textsubscript{i+1}P … [F\textsubscript{i}P … [F\textsubscript{i-1}P … [VP … [V \textbf{PFV} na-pisa] ] ] ] ] ]\hfill internal
    \a<> [CP … [F\textsubscript{i+1}P … [F\textsubscript{i}P … \textbf{PFV} [F\textsubscript{i-1}P … [VP … [V na-pisa] ] ] ] ] ]\hfill external
\xe

Aspect merges in AspP with a constraint that it should comply with the uppermost “aspectual”
affix, such as perfectifising prefix \textit{s-} and imperfectivising suffix \textit{-a-} in (\getref{rus}).
We thus propose that Latvian PP, i. e., usage of perfectivised verbs in present, is in fact
imperfective, like in such contexts in Slavic languages. The question arises what is the reason
for Latvian to allow imperfective aspect to combine with perfectivising prefixes. We have two
options: to posit a null imperfectiviser functionally similar to Russian \textit{-yva-}, or to to allow free
variation of aspect relative to “aspectual” affixes.

Against the first hypothesis stands the fact that in Russian, \textit{-yva-} is mostly used with lexically saturated prefixes and combines quite unproductively with perfectivisers that do not induce meaning change. In the Russian
translation of (\getfullref{pp.ph}), bare imperfective is used instead (\nextx).

\ex<>
    \begingl
        \gla … poet … pish-et / \ljudge*na-pis-yva-jet “Ja”//
        \glb … poet … write-PRS / \Pfv-write-\Ipfv-\Prs{} “I”//
        \glft ‘[In the fourth part of the action ...] the poet crosses out all three “You” and writes “I” instead.’//
    \endgl
\xe

This would lead us to propose that the Latvian imperfectiviser is not only null, but also
unprecedently productive, a poor theoretical choice. We will review the properties
of an “aspectless” analysis and problems related to it. We believe that Latvian data, which is
similar to Russian except for the availability of aspectual interpretations, is a strong argument
towards Tatevosov’s external aspect analysis.
\\~

\textbf{References}

[1] Holvoet, A., Daugavet, A., \& Žeimantienė, V. (2022). The Perfective present in
Lithuanian. Baltic Linguistics, 12 (12), 249-293. https://doi.org/10.32798/bl.925
[2] Horiguchi, D. "Some remarks on Latvian aspect". Valoda: nozīme un forma 4:22-32.
[3] Saulīte, B. et al. Latvian National Corpora Collection – Korpuss.lv
Proceedings of the 13th Language Resources and Evaluation Conference (LREC), 2022,
5123–5129.
[4] Tatevosov, S. Comparative syntax and semantics of Slavic. Course at CreteLing 2023.

\end{sloppypar}
\end{document}