\documentclass[9pt]{beamer}
\usecolortheme[named=black]{structure}
\setbeamertemplate{footline}[frame number]{}
\setbeamertemplate{navigation symbols}{}

\setlength{\parskip}{0.5em}

\usepackage[english]{babel}

% FORMATTING
\usepackage{hyperref}

% TYPOGRAPHY
\usepackage{amsmath}
\usepackage{stmaryrd}
% \usepackage{mathspec}
\usepackage{fontspec}
\usepackage{unicode-math}

\setmainfont{Libertinus Serif}
\setsansfont{Libertinus Sans}
\setmathfont{Libertinus Math}

\usepackage[normalem]{ulem}

% LINGUISTICS 
\usepackage{gb4e,nnext}
\noautomath

\usepackage[glossaries]{leipzig}
\newleipzig{cmpr}{cmpr}{Comparative}
\newleipzig{sprl}{sprl}{Superlative}
\newleipzig{indef}{indef}{Indefinite}


% SUBSTITUTION
\usepackage{hyphsubst}
\usepackage{csquotes}
\MakeOuterQuote{"}

% BIBLIOGRAPHY		    
\usepackage[style=authoryear,url=false,doi=false,isbn=false,eprint=false,date=year]{biblatex}
\addbibresource{../../ref.bib}

% \usepackage{natbib}
% \renewcommand{\bibsection}{~\\\textbf{References}}
% \bibpunct[: ]{[}{]}{;}{a}{}{,}
% \bibliographystyle{rusnat}

% DRAWING
% \usepackage{tikz}
% \usetikzlibrary{shapes.geometric}
% \usetikzlibrary{trees,arrows}
% \usetikzlibrary{positioning}
\usepackage[linguistics]{forest}

%META
\title{Татевосов 2018. Заметки о фальшивой каузативизации \parencite{tatevosov2018zametkifalshivoykauzativizacii}}
\author{Аркадий Шалдов}
\date{06.11.24, ДПВ ПАДы}

\begin{document}

% -------------------------- текстиктекстиктекстик -----------------------------

\begin{frame}
    \titlepage
\end{frame}

\begin{frame}
    \frametitle{Каузативизация в мишарском}

    Нормальный каузатив:

    \begin{exe}
        \ex \gll trener marat-nɤ jeger-t-te\\
            тренер Марат-\Acc{} бегать-\Caus-\Pst\\
            \trans Тренер заставил Марата бегать.
    \end{exe}

    Двойной каузатив:

    \begin{exe}
        \ex \gll trener        kerim-dän        marat-nɤ       jeger-t-ter-de\\
            тренер        Керим-\Abl{}        Марат-\Acc{} бегать-\Caus-\Caus-\Pst\\
            \trans Тренер сделал так, что Керим заставил Марата бегать.
    \end{exe}

    Двойной каузатив со значением одиночного (\emph{фальшивый}):

    \begin{exe}
        \ex \gll trener marat-nɤ jeger-t-ter-de\\
            тренер Марат-\Acc{} бегать-\Caus-\Caus-\Pst\\
            \trans Тренер заставил Марата бегать.
    \end{exe}

    Но не одиночный каузатив без каузативного значения:

    \begin{exe}
        \ex[*]{\gll marat jeger-t-te\\
            Марат бегать-\Caus-\Pst\\
            \trans Марат побегал.}
    \end{exe}

\end{frame}


\begin{frame}
    \frametitle{Теоретические посылки}

    Каузатив вводит новое событие:
    
    \begin{exe}
        \ex $\llbracket -t \rrbracket = \lambda P.\; \lambda x.\; \lambda e.\; \text{causer}(e) = x \land \exists e'\; [P(e') \land \text{cause}(e')(e)]$
    \end{exe}

    Каузатив — реализация \textit{v}
    \begin{itemize}
        \item Внутренний каузатив субкатегоризирует VP
        \item Внешний каузатив субектегоризирует \textit{v}P
    \end{itemize}

    \begin{forest}
        [vP
            [DP\\causer]
            [v'
                [v\\\textsc{cause}]
                [\textit{v}P/VP [{\dots}, roof]]
            ]
        ]
    \end{forest}

\end{frame}

\begin{frame}
    \frametitle{Идея 1: отождествление}

    \begin{itemize}
        \item Два каузатива — два каузирующих подсобытия
        \item Один каузатор
        \item[$\Rightarrow$] Один каузатор в обоих каузирующих подсобытиях
    \end{itemize}

    Формально

    \begin{itemize}
        \item $\llbracket \textsc{caus} \rrbracket = \lambda P.\; \lambda x.\; \lambda e.\; \exists e'\ [P(e') \land \text{causer}(e) = x \land \text{cause}(e')(e)] $ или с дополнительным индивидным аргументом для $P$
    
        \item \textit{v'}-адъюнкт $\llbracket \textsc{refl} \rrbracket = \lambda S.\; \lambda x.\; \lambda e.\; S(x)(x)(e)$
    
        \item $\llbracket \text{тренер Марат-\Acc{} бегать-\Caus-\Caus-\Pst{}} \rrbracket = \lambda e.\; \exists e'\exists e''\; [\textsc{run}(e'')\land\allowbreak \text{agent}(e'') = M \land \text{cause}(e'')(e') \land \text{causer}(e') = T \land \text{cause}(e')(e) \land \text{causer}(e) = T]$ 
    \end{itemize}
    
    `Тренер каузировал себя каузировать Марата побегать.'

\end{frame}

\begin{frame}
    \frametitle{Идея 2: разделение}

    \textcite{pylkkanen2008introducingarguments}: каузирующее подсобытие и аргумент-каузатор вводятся раздельно: [\textsubscript{VoiceP} DP \textsc{voice} [\textsubscript{CauseP} \textsc{cause} [\textsubscript{xp} \dots]]]

    \begin{exe}
        \ex[]{ \gll taroo-ga musuko-o sin-ase-ta\\
            Таро-\Nom{} сын-\Acc{} умирать-\Caus-\Pst{}\\
            \trans У Таро убило сына.}
    \end{exe}
    
    Формально

    \begin{itemize}
        \item $\llbracket \textsc{cause} \rrbracket = \lambda P.\; \lambda e.\; \exists e' [\text{cause}(e')(e) \land P(e)]$
        \item $\llbracket \textsc{voice} \rrbracket = \lambda x.\; \lambda e.\; \text{causer}(e) = x$
        \item идентификация событий (PM для событий с игнорированием верхней валентности)
        \item $\llbracket \text{тренер Марат-\Acc{} бегать-\Caus-\Caus-\Pst{}} \rrbracket = \lambda e.\; \exists e'\exists e''\; [\textsc{run}(e'')\land\allowbreak \text{causer}(e'') = M \land \text{cause}(e'')(e') \land \text{cause}(e')(e)  \land \text{causer}(e) = T]$ 
    \end{itemize}
    
    `Тренер стал каузатором событий, которые повлекли события, которые повлекли бег Марата.'

\end{frame}


\end{document}
