\documentclass[
    9pt,
]{beamer}
\usecolortheme[named=black]{structure}
\setbeamertemplate{footline}[frame number]{}
\setbeamertemplate{navigation symbols}{}
\setbeamertemplate{itemize items}{>}

\setlength{\parskip}{0.5em}

\usepackage[russian,english]{babel}


% TYPOGRAPHY
\usepackage{amsmath}
\usepackage{stmaryrd}
% \usepackage{mathspec}
\usepackage{fontspec}
\usepackage{unicode-math}

\setmainfont{Libertinus Serif}
\setsansfont{Libertinus Sans}
\setmathfont{Libertinus Math}

\usepackage[normalem]{ulem}

\usepackage[dvipsnames]{xcolor}
\let\emph\relax
\DeclareTextFontCommand{\emph}{\color{BrickRed}}


% LINGUISTICS 
\usepackage{expex}
\lingset{aboveglftskip=0ex, belowglpreambleskip=0ex, belowexskip=1ex, aboveexskip=1ex, interpartskip=0ex}
\gathertags
\let\expexgla\gla % dumb unicode-math conflict
\AtBeginDocument{\let\gla\expexgla}

\usepackage[glossaries]{leipzig}
\newleipzig{cmpr}{cmpr}{Comparative}
\newleipzig{sprl}{sprl}{Superlative}
\newleipzig{indef}{indef}{Indefinite}


% SUBSTITUTION
\usepackage{hyphsubst}
\usepackage{csquotes}
\MakeOuterQuote{"}


% BIBLIOGRAPHY		    
\usepackage[style=authoryear,url=false,doi=false,isbn=false,eprint=false,date=year]{biblatex}
\addbibresource{../../ref.bib}


% DRAWING
% \usepackage{tikz}
% \usetikzlibrary{shapes.geometric}
% \usetikzlibrary{trees,arrows}
% \usetikzlibrary{positioning}
\usepackage[linguistics]{forest}


%ALIASES
\ExplSyntaxOn

\NewDocumentCommand{\splist}{ s m m m} {
    #3\mspace{2mu}
    \clist_set:Nn \l_tmpa_clist { #2 }
    \clist_use:Nn \l_tmpa_clist {,\mspace{3mu plus 1mu minus 1mu} \IfBooleanT{#1}{\allowbreak}}
    \mspace{2mu}#4
}

\NewDocumentCommand{\opair}{ s m }{
    \IfBooleanTF{#1}{ \splist*{#2}{\langle}{\rangle} }{ \splist{#2}{\langle}{\rangle} }
}

\NewDocumentCommand{\set}{ s m } {
    \IfBooleanTF{#1}{ \splist*{#2}{\{}{\}} }{ \splist{#2}{\{}{\}} }
}
\ExplSyntaxOff

\newcommand{\denote}[1]{\llbracket #1 \rrbracket}


%META
\title{Функциональные прочтения вопросов\\\parencite{heim2018functionalreadingstypeshifted,haldar2024middlegroundengdahl}}
\author{Аркадий Шалдов}
\date{27 Nov 2024, НИС «Синсем», НИУ ВШЭ}

\begin{document}

% -------------------------- текстиктекстиктекстик -----------------------------

\begin{frame}
    \titlepage
\end{frame}

\begin{frame}
    \frametitle{О чем это?}

    Некоторые вопросы исчисляют по отношениям между индивидами — $\opair{e,e} $
    \ex<>
        — Какого своего родственника не позвала ни одна девочка?\\
        `Какая функция $f_{ee}$ из индивидов в их родственников такова, что ни одна девочка $x$ не позвала $f(x)$'\\
        — Мать / брата / кума / \dots / \#Петю
    \xe    

    Здесь две проблемы
    \begin{itemize}
        \item Во-первых, реляционность ИГ просачивается доверху
        \item Во-вторых, рефлексив не с-коммандуется антецедентом
    \end{itemize}
\end{frame}

\section{Введение}

\begin{frame}
    \frametitle{Значение}

    \ex<>
        Какого своего родственника не позвала ни одна девочка?
    \xe

    \textcite{engdahl1986constituentquestionssyntax}:
    
    \begin{itemize}
        \item $\set{p: \exists f_{\opair{e,e}}\; [\forall x\; [\textsc{родств}(f(x), x) = 1] \land p = (\lambda w_s.\; \neg\exists x\; [\textsc{девочка}(x) = 1 \land \textsc{позвала}(x, f(x) = 1)])]}$
        \item множество пропозиций $p$, для которых есть функция из индивидов в их родственников $f$ такая, что в интенсионале $p$ ни одна девочка не позвала своего $f$
        \item напр. в интенсионале \textit{Ни одна девочка не позвала кума} есть такая функция \textsc{кум}
    \end{itemize}

\end{frame}


\begin{frame}
    \frametitle{Оригинальное решение \parencite{engdahl1986constituentquestionssyntax}}

    
    Что для этого нужно? 
    \begin{itemize}
        \item Семантическое связывание нулевым оператором $E$
                    $\denote{E_y\ \xi}^g = \lambda f_{ee}.\; \forall x.\; \denote{\xi}^{g^{x/y}}(f(x)) = 1$
        \item Полиморфичное \textit{какой} (принимает втч. функции $\opair{e,e}$)\\
                    $\denote{\text{какой}} = \lambda P_{\sigma t}.\; \lambda Q_{\sigma t}.\; \exists x_\sigma [P(x) = 1 \land Q(x) = 1]$
        \item Слоистые следы (из нескольких частей, связываемых в разных местах)
    \end{itemize}

    Структура поверхностная

    \pex<>
        \a<> [$\lambda p$ [какого [$E_y$ [своего$_y$ родственника]] $\lambda f$ [[$Q(p)$] [$\lambda w$ ни одна девочка] [$\lambda x$ [$t_x$ не позвала \emph{$t_{f(x)}$}]]]]]\footnote{$Q$ — Карттуненов прото-вопросный оператор $\lambda p_{st}.\; \lambda q_{st}.\; p = q$}
        \a $\denote{E_y \text{[своего родственника]}} = \lambda f_{ee}.\; \forall z.\; \textsc{родственник}(z)(f(z)) = 1$
    \xe
    
\end{frame}

\section{\cite{heim2018functionalreadingstypeshifted}}
\begin{frame}
    \frametitle{Проблемы особого оператора связывания}

    \textcite{heim2018functionalreadingstypeshifted}:

    \begin{enumerate}
        \item Откуда берутся $\phi$-признаки?
        \ex<>
            Which picture of herself / *himself did no girl submit?
        \xe
        \item Что лицензирует рефлексив?
        \ex<>
            Какого ее / *своего родственника нет на фотокарточке ни одной девочки?
        \xe
    \end{enumerate}

    Все указывает на необходимость с-коммандования. Решение Гейм: \emph{реконструкция}

\end{frame}

\begin{frame}
    \frametitle{Рестриктор in-situ}

    Рестриктор wh-слова \textit{своего родственника} теперь интерпретируется в базовой позиции:

    \ex<>
        [\textsubscript{CP} какого [\textsubscript{TP} [\textsubscript{DP} ни одна девочка] [\textsubscript{VP} не позвала [\textsubscript{DP} своего родственника]]]]
    \xe

    Следствие: \textit{какой} теперь принимает \emph{только один аргумент}

\end{frame}

\begin{frame}
    \frametitle{Композиция по \textcite{heim2018functionalreadingstypeshifted}}

    \only<1>{
        Конверсия следов \parencite{fox2000economysemanticinterpretation}: между копированием и удалением происходит вставление переменной и связывающего оператора:
        \begin{enumerate}
            \item \emph{Изначальная структура:} [$\lambda p$. [[$Q(p)$] [$\lambda w$. Саша позвала [какого студента]]]]
            \item \emph{Копирование:} [$\lambda p$. [[\emph{какого студента}] [[$Q(p)$] [$\lambda w$. Саша позвала [какого студента]]]]]
            \item \emph{Вставление переменной:} [$\lambda p$. [[какого студента] [\emph{$\lambda x$}. [[$Q(p)$] [$\lambda w$. Саша позвала [какого студента \emph{$x$}]]]]]]
            \item \emph{Удаление:} [$\lambda p$. [какого [$\lambda x$. [[$Q(p)$] [$\lambda w$. Саша позвала [студента $x$]]]]]]
        \end{enumerate}
    }
    
    Затайпшифтим:

    \pex<stud>
        \a<full> [$\lambda p$. [какого [$\lambda x$. [[$Q(p)$] [$\lambda w$. Саша позвала [\emph{\textsc{the}} студента [\emph{\textsc{ident}} $x$]]]]]]]
        \a<> $\denote{\textsc{the}} = \lambda P_{et}: \exists! x_\sigma\; [P(x)].\; \iota x_\sigma\; [P(x)]$
        \a<> $\denote{\textsc{ident}} = \lambda x_\sigma.\; \lambda y_\sigma.\; x = y$
        \a<> $\denote{\text{какой}} = \lambda P_{\sigma t}.\; \exists x_\sigma\; [P(x)]$\trailingcitation{одноместное wh-слово}
    \xe
    
    \only<2>{
        Деривация:
    
        \pex<>
            \a<> $\denote{\textsc{ident } x}^g = \lambda y_e.\; g(x) = y$
            \a<> $\denote{\text{студента \textsc{ident }} x}^g = \lambda y_e.\; g(x) = y \land \text{студент}(y)$
            \a<> $\denote{\text{\textsc{the} студента \textsc{ident} } x}^g = g(x)$ е.т.е. \textsc{студент}($g(x)$)
            \a<> $\denote{\text{[$\lambda x$. [[$Q(p)$] [$\lambda w$. Саша позвала [\textsc{the} студента [\textsc{ident} $x$]]]]]}}^g = \lambda x.\; g(p) = \lambda w: \textsc{студент}(g(x)).\; \textsc{позвала}_w(M, g(x))$
            \a<> $\denote{(\text{\getfullref{stud.full})}}\allowbreak = \set{p: \exists x\; [\textsc{студент}(x) \land p = \lambda w.\; \textsc{позвала}_w(M, x)]} \cup \set{\emptyset}$
        \xe
    }

\end{frame}

\begin{frame}
    \frametitle{А функциональные?}

    Позволим \emph{вставку pro на ЛФ} и вставим pro как аргумент $\opair{e,e}$-следа:

    \pex<reltv>
        \a<full> [$\lambda p$. [какого [$\lambda f$. [[$Q(p)$] [$\lambda w$. ни одна девочка] [$\lambda x$. [$t_x$ не позвала [\textsc{the} своего родственника \textsc{ident} [\emph{$f$ pro$_x$}]]]]]]]]
        \a<> $\denote{\text{[\textsc{the} своего родственника \textsc{ident} [$f$ pro$_x$]]}}^g = g(f)(g(y))$ е.т.е. $g(y) \in \text{dom}(g(f)) \land \textsc{родственник}(g(f)(g(y)), g(y))$
        \a<> $\denote{\text{[$\lambda f$. \dots]}} = \lambda f.\; g(p) = \lambda w.: \text{пресуппозиция}.\; \forall y\; [\textsc{девочка}(y) \to \neg \textsc{позвала}(y, f(y))]$
        \a<> $\denote{(\getfullref{reltv.full})} =\linebreak[0] \set{p: \exists f\; [\textcolor{gray}{\forall y\; [\textsc{девочка}(y) \to y\in\text{dom}(f) \land \textsc{родственник}(f(y), y)]} \allowbreak\land p = \lambda w.\; \forall y\; [\textsc{девочка}(y) \to \neg \textsc{позвала}(y, f(y))]]}$
    \xe

\end{frame}

\section{\cite{haldar2024middlegroundengdahl}}

\begin{frame}
    \frametitle{Проблема одноместного \textit{какой} \parencite{haldar2024middlegroundengdahl}}

    В вопросах с функциональными прочтениями возможен эффект избежания реконструкции

    \ex<>
        Какого родственника, которого знает Петя$_i$, он не видел ни у одной девочки в гостях?\\
        `Какая функция $f_{ee}$ из индивидов в их родственников, которых знает Петя, такова, что ни для одной девочки $x$ не верно, что Петя видел у $x$ в гостях $f(x)$?'
    \xe

    \begin{itemize}
        \item Одноместный \textit{какой} предполагает, что весь рестриктор реконструируется
        \item Однако при реконструкции \textit{которого знает Петя} нарушается принцип С теории связывания
    \end{itemize}

    \ex<>
        Какого [\dots{} [\emph{он$_i$} не видел \dots{} [родственника, которого знает \emph{Петя$_i$}]]]
    \xe

\end{frame}

\begin{frame}
    \frametitle{Известная проблема}

    Как объясняют контраст между (\nextx)?

    \pex<>
        \a<> \ljudge*Какую картину Пикассо$_i$ он$_i$ решил купить \sout{какую картину Пикассо}?
        \a<> \ljudge{\textsuperscript{\textsc{ok}}} Какую картину, которую написал Пикассо$_i$, он$_i$ решил купить?
    \xe

    \textcite{sportiche2016neglectdoingaway}:
    \pex
        \a<> \emph{Неучет (Neglect)}: любой материал на любом интерфейсе возможно игнорировать, если это не приводит к провалу
        \a<> \emph{Принцип Полной интерпретации (FI)}: Все синтаксические объекты должны быть интерпретированы хотя бы однажды
        \a<> \emph{Локальное насыщение предикатов}: На LF любой предикат должен быть насыщен аргументами локально
    \xe

    \begin{itemize}
        \item ЛНП: необходимо интерпретировать \textit{Пикассо} (аргумент \textit{картину}, аргумента \textit{купить}) втч. снизу $\implies$ принцип С нарушен
        \item Относительные клаузы не аргументы — их нижние копии могут не учитываться
    \end{itemize}

\end{frame}

\begin{frame}
    \frametitle{Разбиваем рестриктор}

    \begin{itemize}
        \item ИГ необходимо интерпретировать снизу (классические данные)
        \item Относительную клаузу необходимо интерпретировать сверху (данные Халдар)
    \end{itemize}

    \pex<>
        \a<> Какого родственника, которого знает Петя$_i$, он$_i$ не видел ни у одной девочки в гостях?
        \a<> [$\lambda p$. [какого [\emph{которого знает Петя$_i$}] [$\lambda f$. [[$Q(p)$] [$\lambda w$. [[ни одной девочки] [$\lambda y$. [он$_i$ не видел у $t_y$ в гостях [\textsc{the} \emph{родственника} \textsc{ident} [$f pro_y$]]]]]]]]]]
    \xe

    Двухместный \textit{какой}!

\end{frame}

\begin{frame}
    \frametitle{Сразу сложности}

    \pex<>
        \a<> Какого родственника, которого знает Петя$_i$, он$_i$ не видел ни у одной девочки в гостях?
        \a<> [$\lambda p$. [какого [которого знает Петя$_i$] [$\lambda f$. [[$Q(p)$] [$\lambda w$. [[ни одной девочки] [$\lambda y$. [он$_i$ не видел у $t_y$ в гостях [\textsc{the} родственника \textsc{ident} [$f pro_y$]]]]]]]]]]
    \xe

    Какие аргументы у \textit{какого}?

    \begin{itemize}
        \item $\lambda f.\ \dots$ — $\opair{ee, t}$
        \item{} [которого знает Петя] — $\opair{e,t}$
    \end{itemize}

    Типы разные! Оригинальный \textit{какой} Энгдаль такого не поддерживает
    
    \ex<>
        $\denote{\text{какой}} = \lambda P_{\sigma t}.\; \lambda Q_{\sigma t}.\; \exists x_\sigma [P(x) = 1 \land Q(x) = 1]$\trailingcitation{\parencite{engdahl1986constituentquestionssyntax}}
    \xe

\end{frame}

\begin{frame}
    \frametitle{Новый тайпшифтер}

    Какие аргументы у \textit{какого}?

    \begin{itemize}
        \item $\lambda f.\ \dots$ — $\opair{ee, t}$
        \item{} [которого знает Петя] — $\opair{e,t}$
    \end{itemize}
    \ex<>
        $\denote{\text{какой}} = \lambda P_{\sigma t}.\; \lambda Q_{\sigma t}.\; \exists x_\sigma [P(x) = 1 \land Q(x) = 1]$\trailingcitation{\parencite{engdahl1986constituentquestionssyntax}}
    \xe

    У Энгдаль то же самое

    \pex<>
        \a<> $\denote{E_y\ \xi}^g = \lambda f_{ee}.\; \forall x.\; \denote{\xi}^{g^{x/y}}(f(x)) = 1$
        \a<> $\denote{E_y \text{[своего родственника]}} = \lambda f_{ee}.\; \forall z.\; \textsc{родственник}(z)(f(z)) = 1$
    \xe

    Но нам больше не нужно связывать рефлексив

    Поэтому можно проще — тайпшифтер из \textcite{jacobson2002directcompositionalityvariablefree}
    
    \ex<>
        $\denote{E^J} = \lambda P_{et}.\; \lambda f_{ee}.\; \forall x\; [x\in \text{dom}(f) \to P(f(x)) = 1]$
        $\denote{E^J \text{[которого знает Петя]}} = \lambda f_{ee}.\; \forall x.\; [x\in \text{dom}(f) \to \textsc{знает}(f(x))(\text{П}) = 1]$
    \xe

    N. B. Если рестриктора нет, первый аргумент \textit{какой} можно связать контекстным рестриктором

\end{frame}

\begin{frame}
    \frametitle{Последний слайд}

    Сравним два тайпшифтера

    \pex<>
        \a<> $\denote{E_y\ \xi}^g = \lambda f_{ee}.\; \forall x.\; \denote{\xi}^{g^{x/y}}(f(x)) = 1$\trailingcitation{\parencite{engdahl1986constituentquestionssyntax}}
        \a<> $\denote{E^J} = \lambda P_{et}.\; \lambda f_{ee}.\; \forall x\; [x\in \text{dom}(f) \to P(f(x)) = 1]$\trailingcitation{\parencite{haldar2024middlegroundengdahl}}
    \xe

    Первый изменяет функцию присваивания, второй нет
    
    Халдар: тайп-шифтеров (как и языковых выражений), которые изменяют функцию присваивания, в нашей семантике не должно существовать
    
    \ex<>
        \textbf{Гипотеза об изменимости присваиваний}: единственные связыватели переменных, доступные в естественном языке — это $\lambda$-связыватели, присоединенные к структуре при передвижении в А-позицию.
    \xe

\end{frame}

\begin{frame}
    \frametitle{Итоги}

    \begin{itemize}
        \item Функциональные вопросы ставят вопросы, во-первых, о месте интерпретации выражений, а во-вторых — о композиции выражений разных типов
        \item \textcite{engdahl1986constituentquestionssyntax}: рестриктор интерпретируется наверху и связывается семантически (проблемы — эффекты синтаксического связывания)
        \item \textcite{heim2018functionalreadingstypeshifted}: рестриктор интерпретируется снизу, \textit{какой} одноместен (проблемы — избежание реконструкции относительными клаузами)
        \item \textcite{haldar2024middlegroundengdahl}: рестриктор разбивается: основной интерпретируется снизу, относительные клаузы сверху; тайп-шифтер для композиции предикатов над индивидами и функциями
        \item Сложная машинерия: свободная вставка \textit{pro}, два тайпшифтера внизу, один тайпшифтер наверху
        \item И почему мы вообще считаем, что \textit{мать} имеет тип $\opair{e, e}$?
    \end{itemize}

\end{frame}

\begin{frame}
    \frametitle{Ссылки}

    \printbibliography

\end{frame}

\end{document}
